% Options for packages loaded elsewhere
\PassOptionsToPackage{unicode}{hyperref}
\PassOptionsToPackage{hyphens}{url}
\PassOptionsToPackage{dvipsnames,svgnames,x11names}{xcolor}
%
\documentclass[
  letterpaper,
  DIV=11,
  numbers=noendperiod]{scrartcl}

\usepackage{amsmath,amssymb}
\usepackage{iftex}
\ifPDFTeX
  \usepackage[T1]{fontenc}
  \usepackage[utf8]{inputenc}
  \usepackage{textcomp} % provide euro and other symbols
\else % if luatex or xetex
  \usepackage{unicode-math}
  \defaultfontfeatures{Scale=MatchLowercase}
  \defaultfontfeatures[\rmfamily]{Ligatures=TeX,Scale=1}
\fi
\usepackage{lmodern}
\ifPDFTeX\else  
    % xetex/luatex font selection
\fi
% Use upquote if available, for straight quotes in verbatim environments
\IfFileExists{upquote.sty}{\usepackage{upquote}}{}
\IfFileExists{microtype.sty}{% use microtype if available
  \usepackage[]{microtype}
  \UseMicrotypeSet[protrusion]{basicmath} % disable protrusion for tt fonts
}{}
\makeatletter
\@ifundefined{KOMAClassName}{% if non-KOMA class
  \IfFileExists{parskip.sty}{%
    \usepackage{parskip}
  }{% else
    \setlength{\parindent}{0pt}
    \setlength{\parskip}{6pt plus 2pt minus 1pt}}
}{% if KOMA class
  \KOMAoptions{parskip=half}}
\makeatother
\usepackage{xcolor}
\setlength{\emergencystretch}{3em} % prevent overfull lines
\setcounter{secnumdepth}{-\maxdimen} % remove section numbering
% Make \paragraph and \subparagraph free-standing
\makeatletter
\ifx\paragraph\undefined\else
  \let\oldparagraph\paragraph
  \renewcommand{\paragraph}{
    \@ifstar
      \xxxParagraphStar
      \xxxParagraphNoStar
  }
  \newcommand{\xxxParagraphStar}[1]{\oldparagraph*{#1}\mbox{}}
  \newcommand{\xxxParagraphNoStar}[1]{\oldparagraph{#1}\mbox{}}
\fi
\ifx\subparagraph\undefined\else
  \let\oldsubparagraph\subparagraph
  \renewcommand{\subparagraph}{
    \@ifstar
      \xxxSubParagraphStar
      \xxxSubParagraphNoStar
  }
  \newcommand{\xxxSubParagraphStar}[1]{\oldsubparagraph*{#1}\mbox{}}
  \newcommand{\xxxSubParagraphNoStar}[1]{\oldsubparagraph{#1}\mbox{}}
\fi
\makeatother


\providecommand{\tightlist}{%
  \setlength{\itemsep}{0pt}\setlength{\parskip}{0pt}}\usepackage{longtable,booktabs,array}
\usepackage{calc} % for calculating minipage widths
% Correct order of tables after \paragraph or \subparagraph
\usepackage{etoolbox}
\makeatletter
\patchcmd\longtable{\par}{\if@noskipsec\mbox{}\fi\par}{}{}
\makeatother
% Allow footnotes in longtable head/foot
\IfFileExists{footnotehyper.sty}{\usepackage{footnotehyper}}{\usepackage{footnote}}
\makesavenoteenv{longtable}
\usepackage{graphicx}
\makeatletter
\newsavebox\pandoc@box
\newcommand*\pandocbounded[1]{% scales image to fit in text height/width
  \sbox\pandoc@box{#1}%
  \Gscale@div\@tempa{\textheight}{\dimexpr\ht\pandoc@box+\dp\pandoc@box\relax}%
  \Gscale@div\@tempb{\linewidth}{\wd\pandoc@box}%
  \ifdim\@tempb\p@<\@tempa\p@\let\@tempa\@tempb\fi% select the smaller of both
  \ifdim\@tempa\p@<\p@\scalebox{\@tempa}{\usebox\pandoc@box}%
  \else\usebox{\pandoc@box}%
  \fi%
}
% Set default figure placement to htbp
\def\fps@figure{htbp}
\makeatother

\KOMAoption{captions}{tableheading}
\makeatletter
\@ifpackageloaded{caption}{}{\usepackage{caption}}
\AtBeginDocument{%
\ifdefined\contentsname
  \renewcommand*\contentsname{Table of contents}
\else
  \newcommand\contentsname{Table of contents}
\fi
\ifdefined\listfigurename
  \renewcommand*\listfigurename{List of Figures}
\else
  \newcommand\listfigurename{List of Figures}
\fi
\ifdefined\listtablename
  \renewcommand*\listtablename{List of Tables}
\else
  \newcommand\listtablename{List of Tables}
\fi
\ifdefined\figurename
  \renewcommand*\figurename{Figure}
\else
  \newcommand\figurename{Figure}
\fi
\ifdefined\tablename
  \renewcommand*\tablename{Table}
\else
  \newcommand\tablename{Table}
\fi
}
\@ifpackageloaded{float}{}{\usepackage{float}}
\floatstyle{ruled}
\@ifundefined{c@chapter}{\newfloat{codelisting}{h}{lop}}{\newfloat{codelisting}{h}{lop}[chapter]}
\floatname{codelisting}{Listing}
\newcommand*\listoflistings{\listof{codelisting}{List of Listings}}
\makeatother
\makeatletter
\makeatother
\makeatletter
\@ifpackageloaded{caption}{}{\usepackage{caption}}
\@ifpackageloaded{subcaption}{}{\usepackage{subcaption}}
\makeatother

\usepackage{bookmark}

\IfFileExists{xurl.sty}{\usepackage{xurl}}{} % add URL line breaks if available
\urlstyle{same} % disable monospaced font for URLs
\hypersetup{
  colorlinks=true,
  linkcolor={blue},
  filecolor={Maroon},
  citecolor={Blue},
  urlcolor={Blue},
  pdfcreator={LaTeX via pandoc}}


\author{}
\date{}

\begin{document}


\section{Decomposing Inflation}\label{decomposing-inflation}

From the late 1960s through the early 1980s, inflation rose by at least
two to three times in many countries due to preceding high growth, tight
labor markets, and two oil shocks in the 1970s. The ensuing interest
rate hikes, deceleration in economic growth, and the substantial decline
in workers' bargaining power not only made high inflation disappear but
also sometimes caused inflation to stay below the target inflation
pursued by the central banks. In less than one year into the COVID-19
pandemic, the unusually high inflation made a comeback leading to a
cost-of-living crisis for millions of people which in turn led to the
fall of a series of incumbent governments.

Economists point mainly to four factors for high inflation during the
pandemic: supply-chain bottlenecks, fiscal stimuli, concentration in
product markets, and the Russia-Ukraine war with the debate on the role
of each factor continuing.

This project gathers data and makes publicly available the results of a
decomposition analysis that empirical studies have used to link
inflation to wages/salaries and profits.

\subsection{Decomposition Equation}\label{decomposition-equation}

Gross Domestic Product (GDP) can be measured in three different ways
relying on either expenditure, output, or income approach. According to
the income approach, GDP is measured as the sum of the following three
income components:

\begin{enumerate}
\def\labelenumi{\arabic{enumi}.}
\tightlist
\item
  compensation of employees (wages and salaries paid to employees and
  their employers' social contributions) (simply labor costs),
\item
  gross operating surplus (business profits) and gross mixed income
  (profits of the self-employed) (simply profits),
\item
  taxes on production and imports minus subsidies (i.e., net taxes).
\end{enumerate}

Let \(W\), \(S\), and \(T\) denote these (nominal) income components
respectively. Let also \(Y\) and \(Y_r\) denote nominal and real GDP
respectively. The income approach leads to the following identity:

\[Y = W + S + T\]

Dividing both sides by real GDP gives us

\[\frac{Y}{Y_r} = \frac{W}{Y_r} + \frac{S}{Y_r} + \frac{T}{Y_r}\]

or

\[P = w + s + t.\]

\(P\) on the left-hand side of the equation is simply the GDP deflator.
\(w\), the total labor costs divided by real GDP, measures the labor
costs per product and is called unit labor costs. Similarly, \(s\) and
\(t\) denote unit profits and unit net taxes respectively. The change in
the GDP deflator can then be written as

\[\Delta P = \Delta w + \Delta s + \Delta t\]

dividing both sides of which by \(P\) yields

\[\frac{\Delta P}{P} = \frac{\Delta w}{P} + \frac{\Delta s}{P} + \frac{\Delta t}{P}\]

which can be further written as

\[\frac{\Delta P}{P} = \frac{\Delta w}{w}\frac{w}{P} + \frac{\Delta s}{s}\frac{s}{P} + \frac{\Delta t}{t}\frac{t}{P}\]

or

\[\frac{\Delta P}{P} = \frac{\Delta w}{w}\frac{W}{Y} + \frac{\Delta s}{s}\frac{S}{Y} + \frac{\Delta t}{t}\frac{T}{Y}.\]

Finally, we can write

\[\\%\Delta P = \\%\Delta w \frac{W}{Y} + \\%\Delta s \frac{S}{Y} + \\%\Delta t \frac{T}{Y}\]

where \(W/Y\), \(S/Y\), and \(T/Y\) denote labor share, profit share,
and tax share of income respectively.

In words, then, the percentage change in the GDP deflator, i.e.,
domestic inflation, is the weighted sum of the percentage changes in
unit labor costs, unit profits, and unit net taxes where the weights are
the corresponding shares of income. Each additive term in the equation
represents the contribution of a distinct income component per product
to domestic inflation decomposing the latter into three measurable
parts. Empirically, all one needs to do is divide each nominal income
component by real GDP to find the unit income component, compute the
percentage change in it, and then multiply it by the respective income
share for each period.

\subsection{Dashboard}\label{dashboard}

The dashboard, https://bguven.shinyapps.io/inflation-decomposed, makes
available the results of the decomposition analysis which is outlined
above. You can select a country on the sidebar and view the quarterly
contribution of unit labor costs, unit profits, and unit net taxes to
domestic inflation for the whole period for which data are available for
the selected country. You can also focus on the pandemic patterns and
view the decadal averages of the contributions of unit components. The
underlying data behind visuals can be downloaded on the \emph{Tables}
tab of the dashboard.

\subsection{Data Source}\label{data-source}

The decomposition analysis relies heavily on quarterly national accounts
data provided by OECD's data warehouse \emph{OECD Data Explorer}. I
obtained the quarterly \emph{nominal} GDP and components series from
\href{https://data-explorer.oecd.org/vis?fs\%5B0\%5D=Topic,1\%7CEconomy\%23ECO\%23\%7CNational\%20accounts\%23ECO_NAD\%23&pg=40&fc=Topic&bp=true&snb=156&df\%5Bds\%5D=dsDisseminateFinalDMZ&df\%5Bid\%5D=DSD_NAMAIN1@DF_QNA_INCOME&df\%5Bag\%5D=OECD.SDD.NAD&df\%5Bvs\%5D=1.1&dq=Q..AUT..........&to\%5BTIME_PERIOD\%5D=false&lo=5&lom=LASTNPERIODS}{here},
real GDP and deflator series from
\href{https://data-explorer.oecd.org/vis?df\%5Bds\%5D=dsDisseminateFinalDMZ&df\%5Bid\%5D=DSD_NAMAIN1@DF_QNA_EXPENDITURE_INDICES&df\%5Bag\%5D=OECD.SDD.NAD&df\%5Bvs\%5D=1.1&dq=Q............&lom=LASTNPERIODS&lo=5&to\%5BTIME_PERIOD\%5D=false}{here},
and finally Consumer Price Index series from
\href{https://data-explorer.oecd.org/vis?fs\%5B0\%5D=Topic,1\%7CEconomy\%23ECO\%23\%7CPrices\%23ECO_PRI\%23&pg=0&fc=Topic&bp=true&snb=30&df\%5Bds\%5D=dsDisseminateFinalDMZ&df\%5Bid\%5D=DSD_PRICES@DF_PRICES_ALL&df\%5Bag\%5D=OECD.SDD.TPS&df\%5Bvs\%5D=1.0&dq=.M.N.CPI.._T.N.GY+_Z&lom=LASTNPERIODS&lo=13&to\%5BTIME_PERIOD\%5D=false}{here}.

R scripts that extract the series used in the decomposition analysis
from OECD data are all available on the project repository.




\end{document}
